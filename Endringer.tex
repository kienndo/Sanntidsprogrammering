----- PLAN FOR SØNDAG------
kien: (stopp, obstruksjon og flytte funksjoner i backup over på udp)

lage becomeprimary/takeover:
	initialize
	alt som trengs for primary
	lytte til en kanal for en heis 
	generere random tid og tar over som primary hvis den ikke hører om det er noen primary(ingen slave har tatt over masterjobben)
	

---- ETTER IMPLEMENTASJON AV EN HEIS ----------
må fikse obstruksjon og stopp
master slave
backup
nettverksprotokoll (udp)
hall assignment
hva skjer om vi kobler av


---------------HVORDAN SPLITTE MELLOM OM NOE ER CAB ELLER HALL--------------------------------
Jeg splittet Hall og Cab på requests i Elevator-structen
En måte å splitte mellom hall og cab:

func BtnEventSplitter(btnEvent chan elevio.ButtonEvent,
	hallEvent chan elevio.ButtonEvent,
	cabEvent chan elevio.ButtonEvent) {
	for {
		select {
		case event := <-btnEvent:
			if event.Button == elevio.BT_Cab {
				cabEvent <- event
			} else {
				hallEvent <- event
			}
		}
	}
}
---------------------REQUESTS, HALL OG CAB---------------------------------
Requests i Elevator-struct gjøres om til cab og hall slik at vi kan skille mellom, men vi lager en funksjon 
som likestiller dem ved requestfunksjoner i requests.go:
Request := requests_mergeHallAndCab(e.HallRequests, e.CabRequests)

Prøv å skjønne denne: 
func requests_mergeHallAndCab(hallRequests [elevio.N_FLOORS][2]bool, cabRequests [elevio.N_FLOORS]bool) [elevio.N_FLOORS][elevio.N_BUTTONS]bool {
	var requests [elevio.N_FLOORS][elevio.N_BUTTONS]bool
	for i := range requests {
		requests[i] = [elevio.N_BUTTONS]bool{hallRequests[i][0], hallRequests[i][1], cabRequests[i]}
	}
	return requests
}
func FSM_run(drv_buttons chan elevio.ButtonEvent, drv_floors chan int, drv_obstr chan bool, drv_stop chan bool, numFloors int) {

	//Motor direction for testing
	var d elevio.MotorDirection = elevio.MD_Up
	elevio.SetMotorDirection(d)


	// Checks the state of the different input-channels
	go elevio.PollButtons(drv_buttons)
	go elevio.PollFloorSensor(drv_floors)
	go elevio.PollObstructionSwitch(drv_obstr)
	go elevio.PollStopButton(drv_stop)

	// Infinite loop that checks the state of the different input-channels and does something every time it gets a signal
	select {
	case a := <-drv_buttons:
		fmt.Printf("%+v\n", a)
		elevio.SetButtonLamp(a.Button, a.Floor, true)

	case a := <-drv_floors:
		fmt.Printf("%+v\n", a)
		if a == numFloors-1 {
			d = elevio.MD_Down
		} else if a == 0 {
			d = elevio.MD_Up
		}
		elevio.SetMotorDirection(d)

	case a := <-drv_obstr:
		fmt.Printf("%+v\n", a)
		if a {
			elevio.SetMotorDirection(elevio.MD_Stop)
		} else {
			elevio.SetMotorDirection(d)
		}

	case a := <-drv_stop:
		fmt.Printf("%+v\n", a)
		for f := 0; f < numFloors; f++ {
			for b := elevio.ButtonType(0); b < 3; b++ {
				elevio.SetButtonLamp(b, f, false)
			}
		}
	}

}

for f := 0; f < elevio.N_FLOORS; f++ {
			for b := 0; b < elevio.N_BUTTONS; b++ {
				v := elevio.GetButton(elevio.ButtonType(b), f)
				if v != false && prevFloor[f][b] != v {
					fsm.FsmOnRequestButtonPress(f, elevio.ButtonType(b))
				}
				prevFloor[f][b] = v
			}
		}

		{
			// Floor sensor
			g := elevio.GetFloor()
			if g != -1 && g != previous {
				fsm.FsmOnFloorArrival(g)
			}
			previous = g
			

			if timer.TimerTimedOut() == 1 {
				timer.TimerStop()
				fsm.FsmOnDoorTimeout()
			}
		}

		case a := <-drv_obstr:
			//Obstruction
			fmt.Printf("%+v\n", a)
			fmt.Println("OBSTRUUUUUUUCTING!!!!!!!!!!")
			fsm.IsObstructed = a
		}


