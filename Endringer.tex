---------------TING JEG HAR ENDRET SOM ER Å BEMERKE SEG--------------------------------
stor forbokstav på elevator og outputdevice!



---------------HVORDAN SPLITTE MELLOM OM NOE ER CAB ELLER HALL--------------------------------
Jeg splittet Hall og Cab på requests i Elevator-structen
En måte å splitte mellom hall og cab:

func BtnEventSplitter(btnEvent chan elevio.ButtonEvent,
	hallEvent chan elevio.ButtonEvent,
	cabEvent chan elevio.ButtonEvent) {
	for {
		select {
		case event := <-btnEvent:
			if event.Button == elevio.BT_Cab {
				cabEvent <- event
			} else {
				hallEvent <- event
			}
		}
	}
}
---------------------REQUESTS, HALL OG CAB---------------------------------
Requests i Elevator-struct gjøres om til cab og hall slik at vi kan skille mellom, men vi lager en funksjon 
som likestiller dem ved requestfunksjoner i requests.go:
Request := requests_mergeHallAndCab(e.HallRequests, e.CabRequests)

Prøv å skjønne denne: 
func requests_mergeHallAndCab(hallRequests [elevio.N_FLOORS][2]bool, cabRequests [elevio.N_FLOORS]bool) [elevio.N_FLOORS][elevio.N_BUTTONS]bool {
	var requests [elevio.N_FLOORS][elevio.N_BUTTONS]bool
	for i := range requests {
		requests[i] = [elevio.N_BUTTONS]bool{hallRequests[i][0], hallRequests[i][1], cabRequests[i]}
	}
	return requests
}
---------------------EXECUTE REQUESTED HALL ORDER---------------------------------


---------------------HVORDAN FUNGERER EGT FRED SIN KODE---------------------------------
Tar inn alle channels
Definerer en button event som bare skal være ordren som tas inn
Det settes også en timer
Hvorfor settes det så jævlig mange forskjellige timere jesus kristus

Initaliserer en init-heis
Stopper de timerne som jeg ikke skjønner hvorfor timerne er der

Så kommer det en evig while løkke som egt bare er en intialisering av systemet og breaker bare hvis the conditions are right
kan dette gjøres uten while løkke? sånn fsminitbetweenfloors med litt tweak hehe
    de har skrevet at heisen skal starte med stoppedirection(er vel valid det hvis jeg bare justifyer at jeg vil at den bare starter hvor som helst med en MD.stop)
så sjekker de om det kommer inn cab requests og hvis de merker noen cabrequest
    og slår i så fall på lyset der(kan ta bestillinger selv om vi nettopp har startet)
direction settes og elevator sin retning og state settes
Timeren kommer inn igjen, men tror vi dropper den ass for det virker som den bare tar tiden på at dataen sendes

Swith med hvilken tilstand vi er i
    Hvis Idle så skjer ingenting da vi allerede stoppet
    Hvis Moving så sette motor i gang
    Hvis DoorOpen så sett i gang lampen og sett i gang timer
Ferdig med en av disse så ut, ferdig med intialisering
NEI okei den skal ned uansett, eneste grunnen til at den starter med stopp er fordi det kan hende den får en bestillinger
    så casene er basically, får jeg bestilling eller ikke

Så sjekker de egt bare alle channelsene hele tiden gjennom channelsene
    ordre fra hall
    om cab trykkes ned
    hvilken etasje (til indikator)
    case <-doorOpenTimer.C: faktisk litt usikker på hva der her skal bety med tanke på lissom timer men okei, men det er vel om det er en aktiv dørtimer?
    hvis vi får noe fra obstruksjon
    så kommer de to jævla timerne igjen


okei så cluet med timerne er egt en konstant lytting av alle signaler
som definerer hva man ønsker og slå på eller ikke
får du en ordre må du alltid ta i mot og sette den som true i matrisen
---------------------må fikse---------------------------------
dobbelsjekk at logikken gir mening
fikse timer
legge inn hall assigner